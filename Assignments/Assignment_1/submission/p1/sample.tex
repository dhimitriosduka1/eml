%%%%%%%%%%%%%%%%%%%%%%%%%%%%%%%%%%%%%
%%%%%%%%%%%%%%%%%%%%%%%%%%%%%%%%%%%%%
\documentclass{article}

\usepackage[preprint,nonatbib]{neurips_2024}

\usepackage[utf8]{inputenc} % allow utf-8 input
\usepackage[T1]{fontenc}    % use 8-bit T1 fonts
\usepackage{hyperref}       % hyperlinks
\usepackage{url}            % simple URL typesetting
\usepackage{booktabs}       % professional-quality tables
\usepackage{amsfonts}       % blackboard math symbols
\usepackage{nicefrac}       % compact symbols for 1/2, etc.
\usepackage{microtype}      % microtypography
\usepackage{xcolor}         % colors
\usepackage{amsmath}
\usepackage{pgfplots}
\usepackage{caption}  % To add a caption
\usepackage{tcolorbox} % For creating colored or framed boxes
\usepackage{subcaption}

\pgfplotsset{compat=1.18}

%%%%%%%%%%%%%%%%%%%%%%%%%%%%%%%%%%%%%
\title{Assignment \#1\\
  \vspace{2mm}
  \small{Elements of Machine Learning}
  \\
  \vspace{2mm}
  \small{Saarland University -- Winter Semester 2024/25}
}

\author{%
\textbf{Rabin Adhikari} \\
  7072310 \\
  \texttt{raad00002@stud.uni-saarland.de} \\
  \and
  \textbf{Dhimitrios Duka} \\
 7059153 \\
  \texttt{dhdu00001@stud.uni-saarland.de} \\
}

%%%%%%%%%%%%%%%%%%%%%%%%%%%%%%%%%%%%%
\begin{document}
\maketitle

%%%%%%%%%%%%%%%%%%%%%%%%%%%%%%%%%%%%%
\section*{Problem 1 (Parametric and Non-parametric models)}
A parametric model reduces the problem of estimating a function $f$ in finding a set of parameters $\theta$ that best fit the data. When such a model is used, we go through a two-step process. First, we assume the form of the function $f$. This assumption can be as simple as a linear function. Then, we use the training data to train the model, thus estimating the original function $f$. On the other hand, a non-parametric model doesn't make any assumptions for the form of the underlying function $f$. With such models, we aim to find the true form of the function, not an estimation of it. 

In terms of complexity, parametric models exhibit a low complexity since the problem boils down to finding a fixed number of parameters, which doesn't increase with the increase in data points. In contrast, non-parametric models exhibit a high complexity because we aim to find the true form of the underlying function $f$. Therefore, the complexity of this model increases with the increase in the number of data points.

In terms of flexibility, non-parametric models are more flexible than parametric models. This means that non-parametric models can estimate a more complex family of functions without any prior assumptions. In contrast, parametric methods rely on making an assumption on the form of the underlying function $f$, for example, a simple linear assumption, and consequently, limit the family of functions that the model can estimate.

Parametric methods assume a specific form of target data distribution, such as a simple linear relationship in the case of linear models. In contrast, non-parametric models don't make this assumption, such as $k$-NN regression, resulting in a more flexible model.

For both parametric and non-parametric models, training on a small dataset wouldn't result in very good generalization. However, parametric models tend to generalize better since they rely only on a few number of parameters $\theta$. On the other hand, having a large dataset would be beneficial for both model types. Both models, if used properly, would generalize to unseen data.

%%%%%%%%%%%%%%%%%%%%%%%%%%%%%%%%%%%%%
\clearpage

%%%%%%%%%%%%%%%%%%%%%%%%%%%%%%%%%%%%%
% \bibliographystyle{unsrt}
% \bibliography{references}

%%%%%%%%%%%%%%%%%%%%%%%%%%%%%%%%%%%%%
%%%%%%%%%%%%%%%%%%%%%%%%%%%%%%%%%%%%%
\end{document}

%%%%%%%%%%%%%%%%%%%%%%%%%%%%%%%%%%%%%
%%%%%%%%%%%%%%%%%%%%%%%%%%%%%%%%%%%%%